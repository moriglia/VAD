\section{Introduction}

This project is an assignment by Prof. Fanucci to Mancini and Origlia.
The purpose of the project is to practice electronics design using
the hardware description language VHDL and Vivado, after attending the courses
of \textit{Electronics and communications systems} at the Computer Engineering
Master Degree Courses (for Mancini) and \textit{Microelecronics for telecommunications}
at the Telecommunications Engineering Master Degree (for Origlia).

The assignment consists of the design of a Voice Activity Detector (VAD), a
system that takes an audio stream as input and it outputs a logical value that
indicates whether the voice is present or absent.

\subsection*{Specifications}
\begin{itemize}
  \item The input stream is divided in frames of duration
  $T_{frame} = 16\si{\milli\second}$
  \item The sampling rate is $f_s = 16 \si{\kilo\hertz}$
  \item The mean frame energy is $ E_{frame} = \frac{1}{N}\sum_{n=1}^N x^2[n] $
  \item The samples range from -1 to 1: $x[n] \in [-1, 1)$
  \item The mean frame energy is compared to a given threshold:
  $ E_{threshold} = 0.05 $
  \item The input data are represented in 2 complement using 16 bits
  \item The ports of the VAD are:
  \begin{itemize}
    \item \texttt{x}: 16 bits representing the samples
    \item \texttt{FRAME\_START}: it signals the start of the frame by
    a pulse of 1 clock cycle
    \item \texttt{rst\_n}: the system active low reset
    \item \texttt{clk}: the system clock
    \item \texttt{VAD}: the output. High for \textit{voice detected},
    low otherwise.
  \end{itemize}
\end{itemize}
