\section{Problem formal analysis}
The problem we must solve is to tell whether:
\begin{equation*}
  E_{th} \le \frac{1}{N} \sum_{n=1}^N x^2[n]
\end{equation*}
Since $ x[n] \in [-1, 1) $ and it is represented on 16 bits, it can be
written as
\begin{equation*}
  x[n] = z[n]\cdot LSB
\end{equation*}
So $z[n] \in [-2^{15}, 2^{15} - 1]$ and $LSB = \frac{1}{2^{15}}$. Therefor we
can equivalently check that:
\begin{equation*}
  \sum_{n=1}^N |z[n]|^2 \ge \frac{N \cdot E_{th}}{LSB^2} = E_{th}'
\end{equation*}
Since the sampling rate is $f_s$ and the frame duration is $T_{frame}$,
we can easily compute the number of samples per frame:
$N = \frac{T_{frame}}{1/f_s} = 16\si{\milli\second} \cdot 16\si{\kilo\hertz} = 256$.
$E_{th}' = 13743895347$.

\paragraph{Absolute value} of $z > 0$ is $Z$, which is also the representation
of $z$ if it is represented using the 2 complement on 16 bits. We note that
$Z(15) = 0$ because $z > 0$.

If $z < 0$, $|z|$ is represented by $\bar{Z} + 1$, with $Z\in[2^{15}, 2^{16}-1]$.
So $\bar{Z} + 1 \in [1, 2^{15}]$, which requires 16 bits to be represented.
We could approximate the representation of $|z|$ with $\bar{Z}$ for negative $z$,
so the representation would require only 15 bits, as in the case of $z > 0$.

We could hence use only 15 bits to represent $|z|$ for all cases.

\paragraph{Data seriality} can be exploited by using the recursive sum formula
\begin{equation*}
  S[n] = |z[n]|^2 + S[n - 1]
\end{equation*}
$S[N]$ will expand to the sum of the previous equations.
