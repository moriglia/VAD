\section{Implementation}

\subsection{Essential components}
They are:
\begin{itemize}
  \item \texttt{absnetwork}: a combinatorial network that extracts $|z|$,
  its input has 16 bits, its output has 16 bits as well.
  \item \texttt{squarepowernetwork}: a combinatorial network that evaluates
  $|z|^2$ on 32 bits (it is implemented using a multiplier, so for an input of
  16 bits, 32 are required for the result, even though $|z|^2 \le 2^30$ and
  would require no more than 31 bits)
  \item \texttt{accumulator}: performs the recursive sum $S[n] = |z[n]|^2 + S[n - 1]$.
  The maximum sum value is $N \cdot 2^{30} = 16 \cdot 2^{30} = 2^{38}$, requiring 39 bits.
  \item \texttt{comparator}: combinatorial network that compares the output
  of the \texttt{accumulator} with $E_{th}'$, which must be represented on 39
  bits as well in order to be compared to the sum.
\end{itemize}

\subsection{Optimizations}
In order to reduce the number of resources used in the implementation, we can
check for possible optimizations.

\subsubsection{Comparison}
We note that $E_{th}' \simeq 13743895347$ and $\lceil\log_2 E_{th}' + 1\rceil = 34$.
This means that 34 bits would be enough to represent $E_{th}'$. This also implies
that we don't need the accumulator to sum up to more than $2^{34} - 1$, because
if the sum exceeds $2^{34}$, the sum is grater than the threshold.

Moreover we can initialize the initial value of the sum to the overflow value
minus the threshold, so that the overflow will indicate that the threshold has
been exceeded, without using the comparator logic. The initial value of the sum
is $S'[0] = 2^{34} - E'_{th} = 3435973837$ and we compute:
\begin{align*}
  S'[n] &= \bigg| S'[n-1] + |z[n]|^2 \bigg|_{2^{34}}\\
  ovf[n] &= \bigg\lfloor \frac{S'[n-1] + |z[n]|^2}{2^{34}} \bigg\rfloor
\end{align*}
$ovf[n] = 1$ does not imply that $ovf[n + 1] = 1$, but since we are summing
positive integers, the sum will obviously exceed $2^{34}$, so we must remember
that the threshold has been exceeded until the end of the frame.
For this purpose we used a Set-Reset Flip Flop, which is reset at the beginning
of the frame processing and eventually set by $ovf[n]$.

% \subsection{Absolute value}
% The approximation of the absolute value we introduced in the formal analysis
% introduces an error on the energy. This error for negative $x[n]$ can be evaluated as
% \begin{equation*}
%   \epsilon[n] = |z[n]|^2 - \tilde{z}^2[n] = (\bar{Z}[n] + 1)^2 - \bar{Z}^2[n] = 2\bar{Z}[n] + 1
% \end{equation*}
% The worst error we can make corresponds to $\bar{Z}[n] = 2^15 - 1$
% so $\epsilon_{max} = 2^16 -1$

\subsection{Clock and sampling}
The sampling rate is 16\si{\kilo\hertz}, but we can process the data at a faster
rate. If we used a 16\si{\kilo\hertz} clock and after a frame is processed,
another one followed, a clock cycle would be required for \texttt{FRAME\_START}.
So we loose a clock cycle at every frame. In order to avoid so, we could use a
faster clock and divide it, so that one short clock cycle is spent in resetting
everithing at the \texttt{FRAME\_START}, but more clock cycles are
available to process the current input sample before it changes.

In order to perform the clock frequency division we used the
\texttt{frame\_clocker} counter.
The overflow of the counter is used to enable the accumulator, which must
accumulate at a rate equal to the sampling one.
The accumulation must occur at instants which are in the middle of two sample
instants. For this reason, \texttt{frame\_clocker} is reset to a value which is
half way between the initial value and the overflow value.
