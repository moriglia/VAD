\section{Implementation}
TODO:

\begin{itemize}
  \item Schematic
  \item details about chosen implementation
  \item details about DFFE and DFFRE
  \item details about SRFF
  \item details about counter and accumulator
\end{itemize}

\subsection{Clock and sampling}
The sampling rate is 16\si{\kilo\hertz}, but we can process the data at a faster
rate. If we used a 16\si{\kilo\hertz} clock and after a frame is processed,
another one followed, a clock cycle would be required for \texttt{FRAME\_START}.
So we loose a clock cycle at every frame. In order to avoid so, we could use a
faster clock and divide it, so that one short clock cycle is spent in resetting
everithing at the \texttt{FRAME\_START}, but more clock cycles are
available to process the current input sample before it changes.

In order to perform the clock frequency division we used the
\texttt{frame\_clocker} counter.
The overflow of the counter is used to enable the accumulator, which must
accumulate at a rate equal to the sampling one.
The accumulation must occur at instants which are in the middle of two sample
instants. For this reason, \texttt{frame\_clocker} is reset to a value which is
half way between the initial value and the overflow value.
