\section{Testing}

\subsection{Tools}
The following software or languages has been used for testing, simulation or
implementation purposes.

\begin{itemize}
  \item \texttt{ghdl}: HDL syntax, analysis, elaboration and simulation
  \item \texttt{gtkwave}: waveform inspection
  \item \texttt{make}: task automation
  \item \texttt{python}: test case generation
  \item \texttt{Xilinx Vivado}: implementation, performance statistics
\end{itemize}

\subsection{Testing environment}

Commands in this section are supposed to be typed in a bash or dash shell.

\paragraph{The testbench} iteratively takes a string representing a base-10
integer from the standard input. Each integer is interpreted
as one of the input bit vector \texttt{x(15 downto 0)}. The testbench feeds an
instantiation of the VAD with these 16 bit vectors, which will accumulate them.
Every 256 input strings the VAD will produce a binary output. Accordingly,
the testbench will produce an output string representing 0 or 1. The output
string will be streamed to the standard output.

\paragraph{Sample generation} is accomplished by means of the
\texttt{generate\_test.py}, the usage of which is hereby reported:
\footnotesize{
\begin{verbatim}
./generate_test.py [ -s|--sample-file <sample_file> |--stdout ]
    [ -o|--output-file <output_file> | --no-output ]
    [ -n|--sample-count n]
    [ -c|--frame-count k]
    [ -b|--bits b]

    -s,--sample-file <sample_file>      Output the samples to <sample_file>
    --stdout                            Output data to stdin (default)
                                        (overwrites the previous options)
    -o,--output-file <output_file>      Print expected results to <output_file>
    --no-output                         Do not print expected results (default)
                                        (overwrites the previous options)
    -n,--sample-count n=256             Generate n samples per frame
    -c,--frame-count k=1                Generate samples for k frames
    -b,--bits b=16                      Generate samples in C2 on b bits
    -m,--mean-energy 53687091           Use the mean m for the expovariate
                                        distribution
\end{verbatim}
}

The recommended usage of the script is the following:
\begin{verbatim}
./generate_test.py -s samples.in -o samples.out -c 10
\end{verbatim}
This command will create the samples for 10 frames and write them to the
\texttt{samples.in} text file. Each sample will be written on a different line
and it will be in the base-10 representation.

On the other hand, it will output 0 or 1 for each frame of 256 samples, depending
on whether the expected VAD output should be 0 or 1. These outputs, one for each
frame, will be written to the \texttt{samples.out} text file.

In order to check for the VAD behaviour:
\begin{verbatim}
ghdl -r vad_tb --fst=vad_tb.fst < samples.in | diff - samples.out
\end{verbatim}
Where \texttt{vad\_tb} is the compiled testbench. The previous command should not
display anything as long as the architecture of the VAD shows the expected
behaviour.

\paragraph{Waveforms} can be inspected by means of \texttt{gtkwave} after
the previous command:
\begin{verbatim}
gtkwave vad_tb.fst 2> /dev/null &
\end{verbatim}
